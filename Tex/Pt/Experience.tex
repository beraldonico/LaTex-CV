\section{Experiência}
	\begin{itemize}[noitemsep, left=0pt, labelsep = 0pt]
		\item[\linkedlist{Experiencia profissional 4}] { % use \linkedlist{label} to link this item to the next one
			\begin{Experience} % use the Experience environment to create a new experience using especial formatting
				{\textbf{Engenheiro de Desenvolvimento de Software Pleno}, Wex, Remote}
				{06/2025 - presente}
			\end{Experience} 
			 
			\begin{highlights}
				\item Desenvolvendo e mantendo serviços utilizando C\# e .NET para o Health Saving Account Americano.
			\end{highlights} 
		}
		
		\item[\linkedlist{Experiencia profissional 3}] { % use \linkedlist para linkar este item com o próximo
			\begin{Experience} % use o ambiente Experience para criar uma nova experiência com formatação especial
				{\textbf{Analista de TI}, BTG Pactual, São Paulo, SP, Brasil}
				{04/2022 - 06/2024}
			\end{Experience}
			
			\begin{highlights}
				\item Liderei o desenvolvimento de um sistema Full-Stack (C\#, React, MySQL, SQL Server e RabbitMQ) para gerenciamento de relatórios, garantindo conformidade contratual e mitigando riscos financeiros na integração de um novo gestor de fundos no portfólio da empresa.
				\item Liderei a modernização do sistema de relatórios, realizando a transição de Procedures SQL para microserviços, gerando os relatórios 50\% mais rápido e possibilitando a geração em paralelo.
				\item Modernizei sistemas legados utilizando microserviços (C\#, SQL, RabbitMQ), melhorando a eficiência operacional dos sistemas de cálculo de fundos em 40\% além de possibilitar a execução automatizada.
				\item Desenvolvi testes unitários e de integração, garantindo qualidade contínua por meio da integração com pipelines CI/CD e alcançando 100\% de cobertura de código para a lógica de negócio.
				\item Analisei o desempenho do sistema utilizando o Datadog para otimizar a eficiência de custos, mantendo o equilíbrio entre disponibilidade e escalabilidade, resultando em uma redução de 30\% nos custos.
				\item Colaborei com as equipes de negócios para manter e aprimorar sistemas baseados em VB e SQL, garantindo qualidade no serviço e evitando retrabalho para as equipes de negócios.
				\item Gerenciei macros do Excel para 60\% do fluxo de capital offshore, assegurando eficiência e precisão.
			\end{highlights}
		}

		\item[\linkedlist{Experiencia profissional 2}] {
			\begin{Experience}
				{\textbf{Estagiário de TI}, BTG Pactual, São Paulo, SP, Brasil}
				{05/2021 - 04/2022}
			\end{Experience}
			
			\begin{highlights}
				\item Iniciei o desenvolvimento e o uso de testes unitários em C\#, contribuindo para a qualidade do serviço.
				\item Otimizei macros do Excel e MySQL, reduzindo o tempo de processamento em 60\%.
				\item Desenvolvi scripts para mudanças de código, reduzindo o tempo de atualização de links em 80\%.
			\end{highlights}
		}

		\item[\linkedlist{Experiencia profissional 1}] {
			\begin{Experience}
				{\textbf{Estagiário de Automação}, BTG Pactual, São Paulo, SP, Brasil}
				{01/2021 - 05/2021}
			\end{Experience}
			
			\begin{highlights}
				\item Desenvolvi automações com RPA, reduzindo o tempo de execução dos processos em 25 horas diárias.
				\item Desenvolvi um bot de Teams para web scraping e envio automatizado de notícias, otimizando a comunicação interna e reduzindo o tempo de busca por informações relevantes.
			\end{highlights}
		}
	\end{itemize}

	% desenhar linha entre os itens, precisa adicionar uma nova linha para cada novo item
	\vspace{-\baselineskip}
	\begin{tikzpicture}[remember picture,overlay]
		\draw[gray] (Experiencia profissional 1) -- (Experiencia profissional 2);
		\draw[gray] (Experiencia profissional 2) -- (Experiencia profissional 3);
		\draw[gray] (Experiencia profissional 3) -- (Experiencia profissional 4);
	\end{tikzpicture}
	\vspace{-\baselineskip}